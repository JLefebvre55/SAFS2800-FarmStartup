\documentclass{report}
\usepackage{setspace} % Setting line spacing
\usepackage{ulem} % Underline
\usepackage{caption} % Captioning figures
\usepackage{subcaption} % Subfigures
\usepackage{geometry} % Page layout
\usepackage{multicol} % Columned pages
\usepackage{array,etoolbox}
\usepackage{fancyhdr}
\usepackage{enumitem}
\usepackage[table]{xcolor}
\usepackage[toc,page]{appendix}
\usepackage{titlesec} % Section formatting

\usepackage[backend=biber,style=apa,citestyle=authoryear]{biblatex}
\DeclareLanguageMapping{english}{english-apa}
\DeclareFieldFormat{journaltitle}{\textit{#1}}
\DeclareFieldFormat[article]{volume}{\textit{#1}}
\DeclareFieldFormat[misc]{title}{\textit{#1}}
\DeclareFieldFormat[inbook]{title}{\textit{#1}}
\addbibresource{references.bib}

\titleformat{\section}{\normalfont\fontsize{12}{15}\bfseries}{\thesection}{1em}{}
\titleformat{\subsection}{\normalfont\fontsize{12}{15}\bfseries}{\thesubsection}{1em}{}

% Page layout (margins, size, line spacing)
\geometry{letterpaper, left=1in, right=1in, bottom=1in, top=1in}
\setstretch{2}

% Headers
\pagestyle{fancy}
\lhead{ERSC1020 Farm Startup Report}
\rhead{Jayden Lefebvre}

\begin{document}

\begin{titlepage}
    \begin{center}
        \vspace*{1.2cm}

        \textbf{Organizations and Services in New-Entry Agroforestry:\\Resources for Farm Startups and Land-Sharing in Ontario}

        \vspace{2cm}

        Jayden Lefebvre\\

        \vspace{5cm}
        
        Trent University\\
        SAFS 2800H 2025WI\\
        Dr. Helen Knibb\\

        \vfill

        February 14th, 2025
        
    \end{center}
\end{titlepage}

\thispagestyle{plain}
\tableofcontents

\clearpage

% \setstretch{1.5}

\section{Introduction}

%% What is Agroforestry?

\hspace{24pt}Agroforestry is a sustainable agricultural practice wherein trees are planted alongside crops or pastureland. Tree planting has a relatively low investment cost compared to at-scale agriculture, and as such presents a unique opportunity for new-entry farmers to invest in long-term sustainable land management. Partnerships with established farms to extract additional value from the land while limiting interference with existing processes would be mutually beneficial. For example, alley cropping - which involves planting rows of trees between crops - provides shade and buffers against wind and runoff, reducing soil erosion and increasing agrochemical use efficiency while providing additional income streams from lumber, timber, and byproducts (syrup, fruits, etc.) \parencite{benefits}.

% Tree planting presents a myriad of benefits to both farmers and the ecosystem, including:
% \begin{itemize}
%     \item Diversification of income streams (timber, lumber, fruit and nuts);
%     \item Soil health regeneration (nitrogen fixation, erosion prevention);
%     \item Organic carbon sequestration;
%     \item Runoff prevention (buffer zones, water retention);
%     \item Biodiversity improvements (habitat for pollinators, birds, and other wildlife); and
%     \item Ecotourism and land value appreciation.
% \end{itemize}

%% What are the challenges?

\hspace{24pt}Investment in forests presents slow returns, and depends largely on land availability. While existing farmers do have the land available, they may not see the benefits of agroforestry in their lifetime, and are thus less likely to invest. As an alternative, new-entry farmers may be able to create a collaborative land-sharing agreement centered on investment in forest resources to ensure low-risk, long-term results.

\hspace{24pt}This report examines the availability of resources for new-entry farmers investing in agroforestry in Ontario. Furthermore, I will be investigating the intersections between long-term reforestation projects and land-sharing collaboration. Finally, I will address limitations that exist in the currently available supports specifically related to both forest management and land-sharing.

\section{Resources for Farm Startups}

\subsection{Farm Credit Canada}

\hspace{24pt}Farm Credit Canada (FCC) is a crown corporation that provides resources for new and established farmers, including opportunities for funding through repayable loans and grants, which are often limited to purchases of land and equipment. For example, the Young Farmer Loan program provides repayable loans of up to \$2 million to farmers under 40 years of age, even with limited credit \parencite{fcc_youngfarmers}.

\hspace{24pt}FCC also provides a variety of resources for financial planning. For new-entry farmers, they offer Farm Business Plan Tools, which comprises a business plan writing guide with instructions and resources, including a blank business plan template and a sample business plan for reference. These tools make it easy for inexperienced entrepreneurs to easily apply for loans and grants without needing to pay for professional consultation services \parencite{fcc_businessplan}.

\hspace{24pt}In addition, FCC offers Advisory Services, a complimentary consultation program for both new and experienced farmers, including transition planning, for which informal no-nonsense discussion can be an invaluable asset. Their transition planning services include identifying key goals in transition, recommendations on next steps, and finding experts that you can rely on through your transition planning process \parencite{fcc_transition}. The same transition resources could potentially be applied to land-sharing planning.

\subsection{Ontario Soil and Crop Improvement Association}

\hspace{24pt}The Ontario Soil and Crop Improvement Association (OSCIA) offers programs for projects that improve soil health and conservation practices. One such program is OSCIA's Resilient Agricultural Landscape Program (RALP), which provides grants of up to \$3,000 per acre for shrub- and tree-planting initiatives that include at least 4 species and that achieve a final density of at least 700 trees per acre \parencite{oscp_ralp}. Eligible activity expenditures include, beyond site preparation and planting, "costs of third-party technical expertise to support detailed project planning, design, planting/establishment, and on-going maintenance oversight", which would apply directly to young farmers in a land-share planning scenario, especially those from an academic background (i.e. SAFS2800 students).

\subsection{Ontario Woodlot Association}

\hspace{24pt}The Ontario Woodlot Association (OWA) is a non-profit membership organization that provides a wealth of supports for woodlot owners, including educational resources, networking oppoortunities, and assistance with financial planning \parencite{onwoodlot}. Aside from educational workshops, buy-and-sell forums, and newsletters, the OWA also collaborates frequently with outside entities to conserve forested areas and promote sustainable forest management practices (such as the Kawartha Land Trust \parencite{kawartha}).

% \hspace{24pt}The OWA also offers resources specifically for land-sharing. 

\section{Critique and Recommendations}

\subsection{Funding Opportunities}

\hspace{24pt}If funding opportunities with criteria appropriate for agroforestry (i.e. carbon capture, soil health and conservation) are to be leveraged, new-entry farmers would be better served by an increase in the abundance and priority of resources more tailored to long-term and land-share planning such as, for example, a program similar to FedDev Ontario's Advance Payments Program but designed for timber and lumber production. If cash advances were available for woodlot producers, this could enable lowered-risk for farm startups by partnering with existing landowners to invest in forestation projects with advance payment guarantees (i.e. paying salaries and expenses upfront, retirement dividends).

\hspace{24pt}Furthermore, a program similar to the Natural Resources Conservation Service's Environmental Quality Incentives Program (EQIP) in the United States would provide tailored financial and technical assistance to private forest landowners to plan and implement conservation practices that improve soil, water, air, and related natural resources \parencite{eqip}. 

\subsection{Agroforestry Land-Share Planning}

\hspace{24pt}While plenty of resources exist for both agroforestry/woodlot management and land-share planning, there appears to be a gap at the intersection where an NGO or government organization should exist to provide specific, tailored programs to new-entry farmers in Ontario aspiring to collaborate with existing landowners on agroforestry projects. Such an entity would ideally provide a program that connects the two groups, provides structured mentorship and planning resources, and assists in acquiring funding for the project.

% \hspace{24pt}For example, the FCC could offer a land-sharing plan bundle (i.e. templates, samples) similar to their business plan bundle. This could include a sample land-sharing agreement, a template for a land-sharing business plan for new-entry farmers, and online tools and guides to finding and connecting with potential land-sharing partners.

% FCC should offer a land-sharing plan bundle (i.e. templates, samples) similar to their business plan bundle. This could include 

% Funding programs focus largely on the purchasing of land and equipment. If funding opportunities with criteria appropriate for agroforestry (i.e. carbon capture, soil health and conservation) are to be leveraged, new-entry farmers would be better served by resources more tailored to farm succession planning such as, for example, a program similar to the Environmental Quality Incentives Program (EQIP) in the United States. EQIP provides financial and technical assistance to agricultural producers to plan and implement conservation practices that improve soil, water, plant, animal, air, and related natural resources on agricultural land and non-industrial private forestland.

% Additionally, mentorship programs that connect new-entry farmers with experienced agroforestry practitioners could provide valuable hands-on learning opportunities. These programs could facilitate knowledge transfer and help new farmers navigate the complexities of agroforestry management.

% Furthermore, policy recommendations could include the development of tax incentives for retiring farmers who enter into succession plans that prioritize agroforestry. This could encourage more retiring farmers to consider agroforestry as a viable option for their land, thus increasing the adoption of sustainable practices.

% In conclusion, while there are existing resources and funding opportunities for new-entry farmers, there is a need for more targeted support that addresses the unique challenges and benefits of agroforestry. By leveraging appropriate funding criteria, providing mentorship opportunities, and developing supportive policies, new-entry farmers can be better equipped to invest in long-term sustainable land management practices.

\clearpage

% References
\printbibliography

\end{document}